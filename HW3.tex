\documentclass[11pt]{article}

\usepackage{amsmath,amsthm,amssymb,amsfonts,mathrsfs,bbm}
\usepackage[margin=1in]{geometry}

\newcommand{\N}{\mathbb{N}}
\newcommand{\R}{\mathbb{R}}
\newcommand{\tb}{\textbf}
\newcommand{\tn}{\textnormal}

\pagestyle{empty}

\begin{document}

\begin{center}
\noindent \Large{\textbf{MATH 317, Spring 2024 \hfill Problem Set \#3}}\\
Diego Perez
\end{center}

\begin{enumerate}
	\item Let $V$ be a vector space, and let $v_1,v_2,v_3,v_4 \in V$ be such that $\operatorname{span}(v_1,v_2,v_3,v_4) = V$. Prove that $V = \operatorname{span}(v_1-v_2,v_2-v_3,v_3-v_4,v_4)$.
    \\ \textit{ Sol. }
    \begin{itemize}
      \item[] We know that $V = \operatorname{span}(v_1,v_2,v_3,v_4)$. 
      \item[] Then, for any $v \in V$, we can write \begin{center}
           $\alpha_1 v_1 + \alpha_2 v_2 + \alpha_3 v_3 + \alpha_4 v_4= v $\\
           $\alpha_1 v_1 + \alpha_2 v_2 + \alpha_3 v_3 + \alpha_4 v_4 = {\mathbf 0v} \rightarrow$
          $\alpha_1 = \alpha_2 = \alpha_3 = \alpha_4 = 0$
        \end{center}
      \item[] Now, suppose we have scalars  $\alpha, \beta, \gamma, \delta$ 
        such that \begin{align*}
          & \alpha(v_1 - v_2) + \beta(v_2 - v_3) + \gamma(v_3 - v_4) + \delta v_4 = v \\
        \end{align*}
        for some $v \in V$
      \item[] To show linear independence, we need to show that $\alpha = \beta
        = \gamma = \delta = 0$
      \item[] We can rewrite the above equation as \begin{align*}
          & \alpha v_1 + (\beta - \alpha) v_2 + (\gamma - \beta) v_3 + (\delta - \gamma) v_4 = v \\
        \end{align*}
      \item[] Then setting \begin{center}
          $\alpha_1 = \alpha, \alpha_2 = \beta - \alpha, \alpha_3 = \gamma - \beta, \alpha_4 = \delta - \gamma$
        \end{center}
      \item[] Yields \begin{center}
          $\alpha_1 v_1 + \alpha_2 v_2 + \alpha_3 v_3 + \alpha_4 v_4 = v$
        \end{center}
      \item[] Which we know is linearly independent. 
      \item[] Since we have a four linearly independent vectors, 
        we have a basis for $V$. $\Box$
    \end{itemize}

	
    \clearpage
	\item Find a number $t$ such that
	\begin{equation*}
		\begin{bmatrix} 3 \\ 1 \\ 4 \end{bmatrix}, \begin{bmatrix} 2 \\ -3 \\ 5 \end{bmatrix}, \begin{bmatrix} 5 \\ 9 \\ t \end{bmatrix}
	\end{equation*}
	is not linearly independent.
  \begin{itemize}
    \item[] If these are not linearly independent, we must have \begin{align*}
      & \alpha \begin{bmatrix} 3 \\ 1 \\ 4 \end{bmatrix} + \beta \begin{bmatrix} 2 \\ -3 \\ 5 \end{bmatrix} + \gamma \begin{bmatrix} 5 \\ 9 \\ t \end{bmatrix} = \begin{bmatrix} 0 \\ 0 \\ 0 \end{bmatrix} \\
    \end{align*}
    where one of $\alpha, \beta, \gamma$ is non-zero. 
  \item[] This can be rewritten as a augmented matrix \begin{align*}
    & \begin{bmatrix} 3 & 2 & 5 & | & 0 \\ 1 & -3 & 9 & | & 0 \\ 4 & 5 & t & | & 0 \end{bmatrix} \\
    \end{align*}
    $R_1 - 3R_2 \rightarrow R_1 $
    \begin{align*}
    \begin{bmatrix} 0 & 11 & -22 & | & 0 \\ 1 & -3 & 9 & | & 0 \\ 4 & 5 & t & | & 0 \end{bmatrix} \\
    \end{align*}
    $R_3 - 4R_2 \rightarrow R_3$
    \begin{align*}
      \begin{bmatrix}
        0 & 11 & -22 & | & 0 \\ 1 & -3 & 9 & | & 0 \\ 0 & 17 & t - 36 & | & 0
      \end{bmatrix}
    \end{align*}
    $\frac{1}{11}R_1 \rightarrow R_1$
      \begin{align*}
        \begin{bmatrix}
          0 & 1 & -2 & | & 0 \\ 1 & -3 & 9 & | & 0 \\ 0 & 17 & t - 36 & | & 0
        \end{bmatrix}
      \end{align*}
      $\frac{1}{3} R_2 \rightarrow R_2$
        \begin{align*}
          \begin{bmatrix}
            0 & 1 & -2 & | & 0 \\ 1 & -1 & 3 & | & 0 \\ 0 & 17 & t - 36 & | & 0
          \end{bmatrix}
        \end{align*}
    \item[] With this we can write the following system of equations \begin{align*}
      & \beta - 2\gamma = 0 \\
      & \frac{1}{3}\alpha - \beta + 3\gamma = 0 \\
      & 17\beta + (t - 36)\gamma = 0 \\
      \end{align*}
    \item[] When evaluting it, we get \begin{align*}
        & \beta = 2\gamma \\
        & \frac{1}{3}\alpha - 2\gamma + 3\gamma = 0 \rightarrow 
        \alpha = -3\gamma \\
        & 17\beta + (t - 36)\gamma = 0\\
        & 34\gamma + (t - 36)\gamma = 0 \\
        & \gamma(34 + t - 36) = 0 \\
      \end{align*}
      Since we know that $\gamma \neq 0$
      \begin{align*}
        & 34 + t - 36 = 0 \\
        & t - 2 = 0 \\
        & t = 2
      \end{align*}
    \item[] Therefore, the value of $t$ that makes the vectors linearly dependent is $t = 2$. $\Box$

  \end{itemize}
	
  \clearpage
	\item Suppose that $v_1,v_2,v_3,v_4$ is a basis for a vector space $V$. Prove that
	\begin{equation*}
		v_1 + v_2, v_2 + v_3, v_3 + v_4, v_4
	\end{equation*}
	is also a basis.
  \\
  \textit{ Sol. }
  \begin{itemize}
    \item[] We know that $v_1,v_2,v_3,v_4$ is a basis for a vector space $V$. 
    \item[] Therefore, we can write any vector $v \in V$ as a linear 
      combination of $v_1,v_2,v_3,v_4$ as \begin{align*}
        & \alpha_1 v_1 + \alpha_2 v_2 + \alpha_3 v_3 + \alpha_4 v_4 = v \\
      \end{align*}
     \item[] Now, suppose we have scalars $\alpha, \beta, \gamma, \delta$ 
      such that \begin{align*}
      & \alpha(v_1 + v_2) + \beta(v_2 + v_3) + \gamma(v_3 + v_4) + \delta v_4 = \mathbf{0} \\
      \end{align*}
    \item[] To show linear independence, we need to show that $\alpha = \beta
      = \gamma = \delta = 0$
    \item[] Expanding the above equation, we get \begin{align*}
        & \alpha v_1 + \alpha v_2 + \beta v_2 + \beta v_3 + \gamma v_3 + \gamma v_4 + \delta v_4 = \mathbf{0} \\
      & \alpha v_1 + (\alpha + \beta) v_2 + (\beta + \gamma) v_3 + (\gamma + \delta) v_4 = \mathbf{0} \\
      \end{align*}
    \item[] For this equation to hold, we must have \begin{align*}
      & \alpha = 0 \\
      & \alpha + \beta = 0 \\
      & \beta + \gamma = 0 \\
      & \gamma + \delta = 0 \\
      \end{align*}
    \item[] It is trivial to see that the only solution to this system of equations is $\alpha = \beta = \gamma = \delta = 0$. 
    \item[] Therefore, the set $v_1 + v_2, v_2 + v_3, v_3 + v_4, v_4$ is linearly independent. 
    \item[] Since we have a set of four linearly independent vectors, we have a basis for $V$. $\Box$
    
  \end{itemize}

	
  \clearpage
	\item Let $\R_m[x]$ be the vector space of polynomials of degree at most $m$, and let $p_0,\ldots,p_m \in \R_m[x]$ be such that each $p_k$ has degree $k$. Prove that $p_0,\ldots,p_m$ is a basis of $\R_m[x]$.
    \\ \textit{ Sol. }
    \begin{itemize}
      \item[]
      \begin{center}
\fbox{\begin{minipage}{20em}
      \item[] {\bf Note:} $p_0$ is a polynomial of degree 0.
      \item[] Thus, \begin{align*}
          & p_0 = \alpha_{0,0}
        \end{align*}
      where the scalar for the zeroth degree polynomial is attached to the 0th degree variable.
    \item[] As we continue \begin{align*}
      & p_1 = \alpha_{1,0} + \alpha_{1,1}t\\
      & p_2 = \alpha_{2,0} + \alpha_{2,1}t + \alpha_{2,2}t^2\\
      & \dots\\
      & p_m = \alpha_{m,0} + \alpha_{m,1}t + \alpha_{m,2}t^2 + \dots + \alpha_{m,m}t^m\\
    \end{align*}
\end{minipage}}
\end{center}
\item[] Now, suppose we have $\beta_0, \beta_1, \dots, \beta_m$ such that \begin{align*}
    & \beta_0 p_0 + \beta_1 p_1 + \dots + \beta_m p_m = 0
  \end{align*}
\item[] This is equivalent to \begin{align*}
    & (\beta_0 + \alpha_{0,0} + \alpha_{1,0} \dots + \alpha_{m,0}) + 
    \dots + (\beta_m + \alpha{m, 0} + \dots + \alpha_{m,m})t^m = 0 \\
  \end{align*}
\item[] We know that $\alpha_{m,m} \neq 0$ because $p_m$ is a polynomial of degree $m$. 
\item[] Therefore \begin{align*}
    & 0 = \beta_m \alpha_{m,m}\\
    & \beta_m = 0\\
  \end{align*}
\item[] Continuing for all $\beta_k | 0 \leq k \leq m$, we get \begin{align*}
    & \beta_0 = \beta_1 = \dots = \beta_m = 0
  \end{align*}
\item[] Thus the set $p_0, p_1, \dots, p_m$ is linearly independent. 
\item[] Since we have a set of $m+1$ linearly independent vectors, we have a basis for $\R_m[x]$. $\Box$


    \end{itemize}
\end{enumerate}
\end{document}


