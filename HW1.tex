\documentclass[11pt]{article}
%\usepackage{graphicx}    % needed for including graphics e.g. EPS, PS
\usepackage{setspace}
\usepackage{amsfonts}
\usepackage{amsmath}
\usepackage{amssymb}
\usepackage[shortlabels]{enumitem}
\topmargin
-1.5cm % read Lamport p.163
\oddsidemargin
-0.04cm % read Lamport p.163
\evensidemargin
-0.04cm % same as oddsidemargin but for left-hand pages
\textwidth
16.59cm
\textheight
21.94cm
%\pagestyle{empty}       % Uncomment if don't want page numbers
\parskip
7.2pt % sets spacing between paragraphs
\renewcommand{\baselinestretch}{1} % Uncomment for 1.5 spacing between lines
\parindent
0pt % sets leading space for paragraphs

\begin{document}
	\title{Math 317: Theory Of Linear Algebra, Spring 2024\\ Homework Assignment
	1}
	\author{\\ \textbf{Name:}
	Diego Perez}
	\date{}
	\maketitle


  {\bf Problem 1.} Let $x, y \in \mathbb{R}$. Prove that $(-x)(-y) = xy$
  \textit{ Sol. }
  \begin{itemize}
    \item[] We want: \begin{align*}
    & (-x)(-y) = xy \\
    \end{align*}
  \item[] Note: \begin{align*}
    & 0 + x = x \forall x \in \mathbb{R} \\
    & 0x  = 0 \forall x \in \mathbb{R} \\
  \end{align*}
\item[] Therefore it follows that: \begin{align*}
    & (-x)(-y) = (-x)(-y) + 0 \\
    & (-x)(-y) = (-x)(-y) + 0y \\
    & (-x)(-y) = (-x)(-y) + (-x + x)y \\
    & (-x)(-y) = (-x)(-y) + (-x)y + xy \\
    & (-x)(-y) = ((-x)(-y) + (-x)y) + xy \\ 
    & (-x)(-y) = (-x)((-y) + y) + xy\\
    & (-x)(-y) = (-x)(0) + xy\\
    & (-x)(-y) = 0 + xy \\
    & (-x)(-y) = xy. \Box \\
  \end{align*}     
  \end{itemize}
	
  \clearpage
  {\bf Problem 2.} Use induction to prove that\begin{align*}
    & \sum_{k=1}^{n} k = \frac{n(n+1)}{2} \\
  \end{align*}
  \textit{ Sol. }
	\begin{itemize}
    \item[] Base case, $n = 1$ \begin{align*}
        & \sum_{k=1}^{1} k = \frac{1(1+1)}{2} \\
        & 1 = \frac{2}{2} \\
        & 1 = 1 \checkmark \\
    \end{align*} 
  \item[] Assume true for $n = 1, 2, 3 \ldots n$ \begin{align*}
      & \sum_{k=1}^{n} k = \frac{n(n+1)}{2} \\
      & \sum_{k=1}^{n} k + (n+1) = \frac{n(n+1)}{2} + (n+1) \\
      & \frac{n(n+1)}{2} + \frac{(2n) + 2}{2} = \frac{n(n+1)}{2} + \frac{2(n+1)}{2} \\
      & \frac {n^{2} + 3n + 2}{2} = \frac{n^{2} + 3n + 2}{2} \checkmark \\
    \end{align*}
  \item[] Since the statement holds for $n = 1, 2, 3 \ldots n$ and $n+1$,
    then it holds for all $n \in \mathbb{N}$. $\Box$
	\end{itemize}

  \clearpage
  {\bf Problem 3.} 	Prove that if $\alpha \in \mathbb{R}$ and $\mathbf{x} \in \mathbb{R}^n$ are such that
  $\alpha \mathbf{x} = 0$, then $\alpha = 0$ or $\mathbf{x} = \mathbf{0}$
  \textit{ Sol. }
  \begin{itemize}
    \item[] We have two cases, $\alpha = 0$ or $\alpha \neq 0$ in which case $\mathbf{x} = \mathbf{0}$.
    \item[] Case 1: $\alpha = 0$ \begin{itemize}
        \item[] In this case, the proof is trivial since $\alpha = 0$. We know from class that
          multipying any vector by $0$ will result in $\mathbf{0}$.
      \end{itemize}
    \item[] Case 2: $\alpha \neq 0$ \begin{itemize}
        \item[] Since, \begin{align*} & \alpha \neq 0 \\ \end{align*}
      \item[] We know that \begin{align*}
        & \frac{1}{\alpha} \in \mathbb{R} \\
      \end{align*}
    \item[] Thus, \begin{align*}
        & \frac{1}{\alpha} \alpha \mathbf{x}_k = \frac{1}{\alpha} 0 \\
        & (\frac{1}{\alpha} \alpha) \mathbf{x}_k = 0 \\
        & 1 \mathbf{x}_k = 0 \\
        & \mathbf{x}_k = 0 \checkmark \\
      \end{align*}
      \end{itemize}
    \item[] Since any element $\mathbf{x}_k = 0$, then we know that $\mathbf{x} = \mathbf{0}$. $\Box$

      
	\end{itemize}

  \clearpage
  {\bf Problem 4.} Find a square roof $i$; i.e find a complex number $z$ such that \begin{align*} & z^2 = i\end{align*}
  \textit{ Sol. }
	\begin{itemize}
    \item[] Let $z \in \mathbb{C}$ satisfy the proposition
    \item[] Therefore, \begin{align*}
      & z^2 = i \\
      & (a + bi)^2 = i\\
\end{align*}
\item[] Where $a, b \in \mathbb{R}$ and $a = \Re{z}, b = \Im{z}$
\item[] Thus, \begin{align*}
  & (a + bi)^2 = i \\
  & a^2 + 2abi - b^2 = i \\
  & (a^2 - b^2) + 2abi = i \\
\end{align*}
\item[] We need $a^2 - b^2 = 0$ and $2ab = 1$
\item[] This gives: 
\begin{align*}
    &  a = \pm b\\
  \end{align*}

    
        \item[] If $a = -b$, then \begin{align*}
      & 2ab = 1 \\
      & 2b(-b) = 1 \\
      & -2a^2 = 1 \\
      & a^2 = -\frac{1}{2} \\
      & \text{This is not possible since } a \in \mathbb{R} \\
      \end{align*}
    \item[] If $a = b$, then \begin{align*}
        & 2ab = 1\\
        & 2b^2 = 1\\
        & b^2 = \frac{1}{2}\\
        & b = \pm \frac{1}{\sqrt{2}}\\
      \end{align*}
\item[] Thus, \begin{align*}
     & z = a + bi \\
     & z = \pm (\frac{1}{\sqrt{2}} + \frac{1}{\sqrt{2}}i ).\Box \\
   \end{align*}
\end{itemize}

\clearpage
	{\bf Problem 5.} A symmetric matrix is an $n \times n$ matrix such that $A^\top = A$. A
  skew-symmetric matrix is an $n \times n$ matrix such that $A^\top = -A$
\begin{itemize}
  \item[(a)] Prove that \begin{align*}
      & \frac{1}{2}(A + A^\top) \text{ and } \frac{1}{2}(A - A^\top)\\
    \end{align*} are symmetric and skew-symmetric, respectively.\\

    \textit{ Sol. }
    \begin{itemize}
      \item[] Consider \begin{align*}
        & (\frac{1}{2}(A + A^\top))^\top \\
        & (\frac{1}{2}A + \frac{1}{2}A^\top))^\top \\
        & (\frac{1}{2}A^\top + \frac{1}{2}A) \\
      \end{align*}
    \item[] Thus, \begin{align*}
      & (\frac{1}{2}(A + A^\top))^\top =  \frac{1}{2}(A + A^\top) \text{ is symmetric.}\\
    \end{align*}
  \item[] Now, consider \begin{align*}
      & (\frac{1}{2}(A - A^\top))^\top \\
      & (\frac{1}{2}A - \frac{1}{2}A^\top))^\top \\
      & (\frac{1}{2}A^\top - \frac{1}{2}A) \\
    \end{align*}
  \item[] Thus, \begin{align*}
      & (\frac{1}{2}(A - A^\top))^\top = (-1)(\frac{1}{2}(A - A^\top))  \\
    \end{align*}
    \end{itemize}


  \item[(b)] Prove that for every $n \times n$ matrix $A$ there exists unique 
    matrices $S$ and $K$ such that $S$ is symmetric, $K$ is skew-symmetric, and
    $A = S + K$.
    \textit{ Sol. }
    \begin{itemize}
      \item[] We know from $(a)$ that \begin{align*}
          & \frac{1}{2}(A + A^\top) \text{ is symmetric. and}\\
          & \frac{1}{2}(A - A^\top) \text{ is skew-symmetric.}\\
        \end{align*}
      \item[] Thus, let \begin{align*}
          & S = \frac{1}{2}(A + A^\top) \text{ and}\\
          & K = \frac{1}{2}(A - A^\top)\\
        \end{align*}
      \item[] Evaluate, \begin{align*}
          & S + K = \frac{1}{2}(A + A^\top) + \frac{1}{2}(A - A\top) \\
          & S + K = \frac{1}{2}(A + A + A^\top - A^\top)\\
          & S + K = \frac{1}{2}(2A)\\
          & S + K = A \checkmark\\
        \end{align*}
      \item[]Therefore, $S, K$ are a symmetric and skew-symmetric tuple of 
        matrices such that $A = S + K$.
      \item[] Assume that there exists another pair of matrices $S', K'$ such that
        $A = S' + K'$ and $S'$ is symmetric and $K'$ is skew-symmetric.
      \item[] Therefore, 
        \begin{align*}
          & (S + K) - (S' + K') = A - A \\
          & (S + K) - (S' + K') = 0 \\
          & (S - S') + (K - K') = 0 \\
        \end{align*}
      \item[] Observe that $(S - S')$ is symmetric and $(K - K')$ is skew-symmetric: \begin{align*}
        & (S - S')^\top = S^\top - S'^\top \\
        & (S - S')^\top = S - S' \\
        & \text{Same applies for } (K - K') \\
      \end{align*}
    \item[]  Thus, the matrix \begin{align*}
      & (S - S') + (K - K') = 0 \\
      \end{align*} results in a symmetric and skew-symmetric matrix, which only happens in the zero matrix.
    \item[] Therefore, 
      \begin{align*}
        & (S - S') = 0 \text{ and } (K - K') = 0 \\
      \end{align*}
    \item[] Thus, \begin{align*}
      & S = S' \text{ and } K = K' \\
      \end{align*}
    \item[] Finally, we have shown that there exists a unique pair of matrices $S, K$ such that
      $A = S + K$ and $S$ is symmetric and $K$ is skew-symmetric. $\Box$

    \end{itemize}
\end{itemize}
\clearpage
	{\bf Problem 6.} An idempotent matrix is a matrix $A$ such that $A^2 = A$.
  Prove that if $A$ is idempotent, then $I - A$ is also idempotent.\\
  \textit{ Sol. }
  \begin{itemize}
    \item[] Let $A \in \mathbb{R}$ be an idempotent matrix.
    \item[] Evaluate the following: \begin{align*}
        & (I - A)^2 = (I - A)(I - A) \\
        & (I - A)^2 = I^2 - IA - AI + A^2 \\
        & (I - A)^2 = I - 2A + A^2 \\
        & (I - A)^2 = I - 2A + A \\
        & (I - A)^2 = I - A \\
      \end{align*}
    \item[] Thus, $(I - A)$ is idempotent. $\Box$
  \end{itemize}
  \end{document}
